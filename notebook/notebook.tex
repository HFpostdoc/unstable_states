%%%NOTE: for help with latex symbols look here http://mirror.unl.edu/ctan/info/symbols/comprehensive/symbols-a4.pdf.
\documentclass[12pt]{article}
\usepackage{color}
\usepackage{cite}
\usepackage{geometry}                % See geometry.pdf to learn the layout options. There are lots.
%\usepackage{pdflscape}        %single page landscape
                                %mode \begin{landscape} \end{landscape}
\geometry{letterpaper}                   % ... or a4paper or a5paper or ... 
%\usepackage[parfill]{parskip}    % Activate to begin paragraphs with an empty line rather than an indent
\usepackage{multicol} % \begin{multicols}{number of columns} \end{multicols}
% \usepackage{lineno} %add lines numbers 
\usepackage{graphicx}
\usepackage{amssymb}
\usepackage{Sweave}
\newcommand{\etal}{\textit{et al.}}
\usepackage{hyperref}  %\hyperref[label_name]{''link text''}
                       %\hyperlink{label}{anchor caption}
                       %\hypertarget{label}{link caption}
\linespread{1.5}

\title{Unstable States: Community dynamics in a chaotic world}
\author{M.K. Lau}
%\date{}                                           % Activate to display a given date or no date

\begin{document}
\maketitle
%\linenumbers %%add line numbers

%\setcounter{tocdepth}{3}  %%activate to number sections
%\tableofcontents

%\thispagestyle{empty}
%\setcounter{page}{0}
%\setcounter{secnumdepth}{-1}  %activate to start numbering from one
%on the second page

\section{19 Jun 2014}

The pitcher plant respiration model:

Terms:

\begin{itemize}
\item t = time
\item x = oxygen
\item A = environmental factor promoting oxygenation (i.e. PAR = light)
\item f(w,x) = loss/decay of oxygen 
\item g(x) = recovery of oxygen (augmented by mineralized nutrients)
\item w = prey mass
\end{itemize}

\begin{equation}
  \frac{dx}{dt}  = A - f(w,x) + g(x)
\end{equation}

\begin{equation}
  x_{t+1} = a_t * sin(2 \pi f t) -  (m + a_t \frac{w_{t-1}}{K_w +
    w_{t-1}}) + D_t(x_t,a^{\prime}_t)
\end{equation}



\begin{Schunk}
\begin{Sinput}
>   A <- function(a0,f){
+     
+   }
\end{Sinput}
\end{Schunk}

\begin{equation}
  
\end{equation}

\begin{verbatim}
Here is a link to the pitcher-plant tipping point dropbox. To
start, go to ../Model sensitivity analysis and second paper/data/ and
read the metadata.csv file. The various raw and average files are the
ones to think about plotting. We can talk about it later today.
\end{verbatim}

\section{18 Jun 2014}

Question: Are transitions between chaotic states detectable?

Question: Are there classes of models that possess detectable chaotic
transition warning signals?

Modeling Issues:
\begin{enumerate}
\item a in S2.2 can be confused with a in S2.1
\item What is the exponent in S2.2?
\item Why is b (S2.2) in days rather than minutes?
\item Why is w(t-1) used in S2.3?
\end{enumerate}

\begin{Schunk}
\begin{Sinput}
>   ##Initial stab at the spo2 model
>   a0 <- 10
> f <- 1/(60*24) #total minutes in a day
> t <- rep((1:(60*24)),11)
> At <- a0 * sin(2*pi*f*t)
> At[t>720] <- 0
> plot(At~I(1:length(t)),type='l')
> a <- 20
> b <- 4
> wt <- 20
> wt <- a*exp(1)^(-b*wt)
> spo2 <- function(x,a,f,t,m,w,Kw,s,d){
+   At <- a0 * sin(2*pi*f*t)
+   
+   
+ }
> 
\end{Sinput}
\end{Schunk}



\section{17 Jun 2014}

Chatted with Aaron

The data don't support a general pattern of alternative states.

If systems are inherently chaotic or stochastic, then what?

The data:
\begin{itemize}
\item Climate isn't stable
\item Fossil records don't show stability
\item Tropical systems don't show stability
\item Even temperate systems break the rule of stability
\end{itemize}

Two goals:
\begin{enumerate}
\item Can we tell when we transition between states?
\item Is there an alternative philosphy? Mathematical framework?
\end{enumerate}

Run simulation increasing r with variance over the chaos threshold

\begin{Schunk}
\begin{Sinput}
>   ##hold ni constant
>   ##hold sd constant at 0.1
>   ##record r
>   ##record n
>   ##record ews
>   source('../src/unstable_states.R')
> library(earlywarnings)
> cb8.16 <- 2.57 #choatic boundary, 8-16 cycles
> ri <- (cb8.16)-(0.02/2)
> rf <- (cb8.16)+(0.02/2)
> dmc <- list()
> ews <- list()
> mcn <- 100
> for (i in 1:mcn){
+   print(i)
+   dmc[[i]] <- disrupt.mc(N=10,sd=0.01,ri=ri,rf=rf,dump=TRUE)
+   ews[[i]] <- generic_ews(dmc[[i]]$N)
+   dev.off()
+ }
> ## dput(dmc,'../results/dmc.out')
> ## dput(ews,'../results/ews.out')
> ###
> ###
> ###
> dmc <- dget('../results/dmc.out')
> ews <- dget('../results/ews.out')
> ##
> yl <- apply(do.call(rbind,ews),2,min)
> yu <- apply(do.call(rbind,ews),2,max)
> ##
> par(mfrow=c(2,(ncol(ews[[1]])-1)/2))
> for (i in 2:ncol(ews[[1]])){
+   for (j in 1:length(ews)){
+     if (j==1){
+       plot(ews[[j]][,i]~ews[[j]][,1],ylim=c(yl[i],yu[i]),
+            ylab=colnames(ews[[1]])[i],xlab='t',
+            type='l',lwd=0.25)
+     }else{
+       lines(ews[[j]][,i]~ews[[j]][,1],lwd=0.25)
+     }
+   }
+ }
> ###Average plots
> ews. <- do.call(rbind,ews)
> t <- do.call(rbind,dmc)$t
> r <- do.call(rbind,dmc)$r
> r. <- r
> r[r.>=cb8.16] <- 2
> r[r.<cb8.16] <- 1
> r <- r[t>=ews[[1]][,1][1]]
> #pairs(ews.,cex=0.05,pch=19,col=r)
> par(mfrow=c(1,1))
> plot(ews.[,c(3,4)],pch=19,col=r,cex=(0.01+(0.5*(ews.[,1]/max(ews.[,1])))))
> unique(ews.[ews.[,3]>30.15&ews.[,4]>0.185,1])
> ews. <- apply(ews.,2,function(x,t) tapply(x,t,mean),t=ews.[,1])
> pairs(ews.,cex=0.10,pch=19)
> 
\end{Sinput}
\end{Schunk}


Reading Hastings
Reading Sheffer
Reading Dakos

\section{16 Jun 2014}

\begin{itemize}
\item the distribution of the average of r is uniform
\item ensemble distribution is normal
\item EWS stats not correlated between 8to16 and 16to8
\item Phase (Ni to Nf) spaces for EWS correlations 
\item EWS stats intercorrelations show correlations and break points
\end{itemize}

\begin{Schunk}
\begin{Sinput}
> ###Run repeated simulations for ews time series
> source('../src/unstable_states.R')
> library(earlywarnings)
> cb8.16 <- 2.57 #choatic boundary, 8-16 cycles
> ri <- (cb8.16)-(0.02/2)
> rf <- (cb8.16)+(0.02/2)
> ##Visualizing the error in r
> r8.16 <- list()
> r16.8 <- list()
> for (i in 1:138){
+   print(i)
+   r8.16[[i]] <- disrupt.mc(N=i,sd=0.01,ri=ri,rf=rf,dump=TRUE)$r
+   r16.8[[i]] <- disrupt.mc(N=i,sd=0.01,ri=rf,rf=ri,dump=TRUE)$r
+ }
> rmu8.16 <- apply(do.call(rbind,r8.16),2,mean)
> rmu16.8 <- apply(do.call(rbind,r16.8),2,mean)
> par(mfrow=c(1,2))
> plot(density(rmu8.16),main='',xlab='r')
> for (i in 1:length(r8.16)){
+   lines(density(r8.16[[i]]),col='grey',lwd=0.5)
+ }
> abline(v=cb8.16,lty=2);abline(v=mean(rmu8.16),lty=2,col='darkgrey')
> plot(density(rmu16.8),xlab='r',main='')
> for (i in 1:length(r16.8)){
+   lines(density(r16.8[[i]]),col='grey',lwd=0.5)
+ }
> abline(v=cb8.16,lty=2);abline(v=mean(rmu16.8),lty=2,col='darkgrey')
> ###Determine average threshold
> ##What is the point at which rbar has crossed the threshold?
> ##Directionality depends on direction of r
> rt8.16 <- (1:length(rmu8.16))[rmu8.16>=cb8.16][1]
> rt16.8 <- (1:length(rmu16.8))[rmu16.8<=cb8.16][1]
> rt8.16
> rt16.8
> ###EWS stats
> stats8.16 <- dget(file='../results/stats816.rdata')
> stats16.8 <- dget(file='../results/stats168.rdata')
> ###
> stats8.16 <- na.omit(do.call(rbind,stats8.16))
> stats16.8 <- na.omit(do.call(rbind,stats16.8))
> ###
> stats8.16 <- stats8.16[1:min(c(nrow(stats8.16),nrow(stats16.8))),]
> stats16.8 <- stats16.8[1:min(c(nrow(stats8.16),nrow(stats16.8))),]
> ###
> par(mfrow=c(2,ncol(stats8.16)/2),
+     mai=c(0.25,0.01,0.25,0.01))
> for (i in 1:ncol(stats8.16)){
+   plot(density(stats8.16[,i]),
+        main=colnames(stats8.16)[i],
+        xlim=c(min(c(stats8.16[,i],stats16.8[,i])),
+          max(c(stats8.16[,i],stats16.8[,i]))),
+        xaxt='n',yaxt='n',bty='n')
+   lines(density(stats16.8[,i]),lty=2)
+ }
> ###
> par(mfrow=c(2,ncol(stats16.8)/2))
> for (i in 1:ncol(stats8.16)){
+   plot(stats16.8[,i]~stats8.16[,i],
+        xlab=paste(colnames(stats8.16)[i],'8.16'),
+        ylab=paste(colnames(stats16.8)[i],'16.8'))
+   abline(lm(stats16.8[,i]~stats8.16[,i]))
+ }
> ###
> par(mfrow=c(4,ncol(stats8.16)/2),
+     mai=c(0,0,0,0))
> for (i in 1:ncol(stats8.16)){
+   plot(stats8.16[1:(nrow(stats8.16)-1),i]~stats8.16[2:(nrow(stats8.16)),i],
+        type='l',xaxt='n',yaxt='n',bty='n')
+ }
> for (i in 1:ncol(stats16.8)){
+   plot(stats16.8[1:(nrow(stats16.8)-1),i]~stats16.8[2:(nrow(stats16.8)),i],
+        type='l',col='darkgrey',xaxt='n',yaxt='n',bty='n')
+ }
> ###Ensemble N~stats
> pairs(data.frame(Ni=(1:nrow(stats8.16)),stats8.16),pch=19,cex=0.10,col='black')
> pairs(data.frame(Ni=(1:nrow(stats16.8)),stats16.8),cex=0.10,col='black')
> 
\end{Sinput}
\end{Schunk}

\begin{Schunk}
\begin{Sinput}
> library(earlywarnings)
> set.seed(1)
> drmc8.16 <- disrupt.mc(sd=0.01,ri=ri,rf=rf,dump=TRUE)
> set.seed(1)
> drmc16.8 <- disrupt.mc(sd=0.01,ri=rf,rf=ri,dump=TRUE)
> ews8.16 <- generic_ews(drmc8.16$N)
> ews16.8 <- generic_ews(drmc16.8$N)
> stats8.16 <- cor(ews8.16,method='ken')
> stats16.8 <- cor(ews8.16,method='ken')
> 
> 
\end{Sinput}
\end{Schunk}

\begin{Schunk}
\begin{Sinput}
>   ##Re-doing noise in r shifting up and down across 8-16
> source('../src/unstable_states.R')
> cb8.16 <- 2.57 #choatic boundary, 8-16 cycles
> ri <- (cb8.16)-(0.02/2)
> rf <- (cb8.16)+(0.02/2)
> set.seed(1)
> drmc8.16 <- disrupt.mc(sd=0.01,ri=ri,rf=rf,dump=TRUE)
> set.seed(1)
> drmc16.8 <- disrupt.mc(sd=0.01,ri=rf,rf=ri,dump=TRUE)
> par(mfrow=c(3,2))
> plot(drmc8.16$r,xlab='time',ylab='r')
> abline(h=cb8.16,col=2)
> plot(drmc16.8$r,xlab='time',ylab='r')
> abline(h=cb8.16,col=2)
> plot(drmc16.8$N,xlab='time',ylab='N')
> plot(drmc8.16$N,xlab='time',ylab='N')
> plot(drmc16.8$N~drmc16.8$r,xlab='r',ylab='N',type='l')
> plot(drmc8.16$N~drmc8.16$r,xlab='r',ylab='N',type='l')
> ###Phase space
> par(mfrow=c(1,2))
>                                         #plus n time steps
> for (n in 25:50){
+   plot(drmc16.8$N[(1:(length(drmc16.8$N)-n))],drmc16.8$N[((n+1):(length(drmc16.8$N)))],
+        xlab='N',ylab='N+1',type='l')
+   plot(drmc8.16$N[(1:(length(drmc8.16$N)-n))],drmc8.16$N[((n+1):(length(drmc8.16$N)))],
+        xlab='N',ylab='N+1',type='l')
+ }
> #EWS
> library(earlywarnings)
> ews8.16 <- generic_ews(drmc8.16$N)
> ews16.8 <- generic_ews(drmc16.8$N)
> 
\end{Sinput}
\end{Schunk}

Summary to date (going back in time):

\begin{itemize}
\item At the 8-16 cycle threshold, error in r leads to early, sudden shifts
\item Slow ramping can also be seen visually
\item Sudden jumps across cycle boundaries can be seen visually
\end{itemize}



%% %%Activate for bibtex vibliography
%% \cite{goossens93}
%% \bibliographystyle{plain}
%% \bibliography{/Users/Aeolus/Documents/bibtex/biblib}


\end{document}  


